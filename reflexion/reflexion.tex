\documentclass[12pt, a4paper, twoside]{article}
\usepackage{../labreport}
\usepackage{pdfpages}
\usepackage{tikz}


\setlabreportopts[authors={Nandor Kovacs \& Céline Schuster},
    title={Mathematisches Pendel},
    subtitle={Untersuchung der Eigenschaften eines Mathematischen Pendels},
    date={\today},
    labdate={12. April 2022}
]

\begin{document}

\section{Einleitung}
\section{Theorie}
\section{Experiment}
\subsection{Versuch 1}
Bei diesem Versuch haben wir einen Glasquader auf einene Blatt Papier gestellt.
Auf der einen Seite haben wir einen geraden Objekt schräg hingestellt.
Auf der anderen Seite haben wir auch einen geraden Objekt hingestellt, 
aber so dass es mit dem vorherigen Objekt in einer Linie ist wenn man durch den Quader schaut.
Dann haben wir die Einfall, Brechungs, und Ausfallswinkel gemessen.



\subsection{Versuch 2}
\section{Aufgaben}
\section{Fazit}
\section{Reflexion}

\end{document}
