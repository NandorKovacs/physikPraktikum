\documentclass[12pt, a4paper, twoside]{article}
\usepackage{labreport}
\usepackage{pdfpages}

\setlabreportopts[authors={Nandor Kovacs \& Céline Schuster},
    title={Mathematisches Pendel},
    subtitle={Untersuchung der Eigenschaften eines Mathematischen Pendels},
    date={\today},
    labdate={4. März 2022}
]

\begin{document}
\maketitlepage

\includepdf[pages={1}]{aufgabenstellung.pdf}

\section{Einleitung}
Wir alle kennen Pendeluhren.
Pendeluhren benutzen statt Quartz stückchen, oder Signale von einer Radiouhr eine Pendel um die Zeit zu verfolgen.
Das muss heissen, das die Periodenlängen von Pendeln berechenbar sind.
Falls die Periode von dem Ausschlagwinkel abhängt, würde das aber recht kompliziert sein zum berechnen, also vermuten wir dass das nicht der Fall ist.

\section{Theorie}
Eine Pendel hat folgende Parameter:
\begin{list}{-}{}
  \item \emph{l} = die Länge der Pendel
  \item \emph{g} = das Gewicht des Pendelgewichts. Dies ist gleich zur länge der Schnur plus die halbe Dicke des Gewichts.
  \item \emph{$\alpha$} = die auslenkung beim starten der Pendel in winkelgrad
\end{list}

Die Formel für die länge einer Periode in Sekunden lautet
$$T=2\pi\cdot\sqrt{\frac{l}{g}}$$
\cite{FoTa}

\section{Experiment}
\begin{figure} [ht]
  \centering
  \includegraphics[width=0.5\textwidth]{Experimentaufbau.png}
  \caption{Skizze Experimentaufbau}
  \label{fig:experimentaufbau}
\end{figure}

Unser Experiment ist einfach aufgebaut. 
Es wird ein Gewicht an einem Faden Aufgehängt. 
Sie wird ausgelenkt, losgelassen, und die Zeit der Perioden werden gemessen wenn das Gewicht die Mitte kreuzt. 
Wir lassen ausserdem 5mal das Gewicht pendeln bevor wir es messen, damit ungenauigkeiten ausgependelt werden.

Die länge $l$ messen wir mit einem Massband auf 0.5cm genau und $\alpha$ messen wir mit einer Winklerscheibe auf 2\textdegree genau. Die Zeit messen wir mit einer Stoppuhr.

Um den Messfehler der Zeitmessung zu erfassen machen wir 20 Messungen und finden die maximale Abweichung zum Mittel der Messungen.
\datatable{2}{Messungen zu Aufgabe a}{Messung Nr. & Schwingungsdauer $T$ in s}{Messung_a.csv}

\begin{align*}
  \overline{T} & = \frac{1}{n} \sum_{i=1}^{n} T_{i}=\frac{1}{n}\left(T_{1}+\cdots+T_{n}\right) \\
               & =\frac{32.5s}{20}                                                             \\
               & =1.625s                                                                       \\
  \\
  \Delta T     & = T_{max} - \overline{T}                                                      \\
               & = 1.8s - 1.625s                                                               \\
               & = 0.175s                                                                      \\
               & = 0.2s
\end{align*}

\vfill\pagebreak

\section{Aufgaben}
\subsection{Abhängigkeit Schwingungsdauer und Pendelmasse}
\datatable{2}{Messungen zu Aufgabe b}{Pendelmasse $m$ in g \textdegree & Schwingungsdauer $T$ in s}{Messung_b.csv}

\datadiagram[x label=Pendelmasse $m$ i g,
  x data=m,
  x error=0.1,
  y label=Schwingungsdauer $T$ in s,
  y data=T,
  y error=0.2
]{Messung_b.csv}

Da die Differenzen zwischen den Schwingungsdauern bei den verschiedenen Pendelmassen kleiner als die Messfehler sind können wir annehmen, dass die Pendelmasse keinen Einfluss auf die Schwingungsdauer hat.

\subsection{Abhängigkeit Schwingsungsdauer und Amplitude}
\datatable{2}{Messungen zu Aufgabe c}{Amplitude $\alpha$ in \textdegree & Schwingungsdauer $T$ in s}{Messung_c.csv}

\datadiagram[x label=Amplitude $\alpha$ in \textdegree,
  x data=alpha,
  x error=2,
  y label=Schwingungsdauer $T$ in s,
  y data=T,
  y error=0.2
]{Messung_c.csv}

Auch hier sind die Differenzen kleiner als der Messfehler, daher können wir auch annehmen, dass die Schwingungsdauer von der Maplitude unabhängig ist.

\subsection{Schwingungsdauer als Funktion der Pendellänge}
\datatable{2}{Messungen zu Aufgabe d}{Pendellänge $l$ in cm & Schwingungsdauer $T$ in s}{Messung_d.csv}
\datadiagram[x label=Pendellänge $l$ in cm,
  x data=l,
  x error=0.1,
  y label=Schwingungsdauer $T$ in s,
  y data=T,
  y error=0.2
]{Messung_d.csv}

\section{Fazit}
\section{Reflektion}
\section{Referenzen}
\printbibliography[heading=none]
\section{Anhang}
Veruschsanleitung und Originalprotokoll vom \labdate

\end{document}