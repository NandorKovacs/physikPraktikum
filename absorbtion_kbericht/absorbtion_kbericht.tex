\documentclass[12pt, a4paper, twoside]{article}
\usepackage{pdfpages}
\usepackage{../labreport}

\setlabreportopts[authors={Nandor Kovacs \& Céline Schuster},
    title={Mathematisches Pendel},
    subtitle={Untersuchung der Eigenschaften eines Mathematischen Pendels},
    date={\today},
    labdate={10. März 2022}
]

\begin{document}
  \section{Versuch 2} 
  \subsection{}
  \datadiagram{  
    enlargelimits=0.2,
    ylabel={Event Anzahl},
    xlabel={Bleiplattendicke in mm}
  }
  {
    \addplot+[color=red] table[x expr=\thisrow{mm}, y expr=\thisrow{e} - 38, col sep=comma] {messungen.csv};
    \addplot+[color=yellow, domain=0:21.5, mark=none]{347.5};
  }

  Bei 0mm beträgt die Anzahl von Events im Zeitrahmen 695.
  Die Hälfte davon ist 347.5. Die gelbe Linie zeichnet diesen Wert ein.
  Nach unseren Daten ist der Halbwertswert etwa bei 12.5mm. 

  \subsection{}
  Strahlung wird beim Zusammenstoss mit Materialien teilweise absorbiert und umgelenkt.
  Je dicker das Hinderniss, desto mehr von der Strahlung wird absorbiert und abgeschirmt.
  Eine Schicht des Materials blockiert einen bestimmten Anteil der Strahlung.
  Das ist der Grund weswegen Funktion am Anfang viel Steiler ist. 
  Beim 1. Millimeter wird der ganze Anteil blockiert, beim 2. Millimeter ist dieser Anteil aber schon kleiner, da es der Anteil von der Strahlung ist, die noch "Überlebt" hat.
  \section{Versuch 3}
  \subsection{}
  Durchschnittlich wird man in der Schweiz 4.2mSv pro Jahr ausgesetzt. 
  Davon sind:
  \begin{itemize}
    \item 1.6mSv Strahlung durch Radon in Wohnräumen
    \item 1.2mSv Strahlung durch medizinische Anwendungen
    \item 0.45mSv terrestrische Strahlung
    \item 0.4mSv innere Bestrahlung
    \item 0.35mSv kosmische Strahlung
    \item 0.2mSv übrige Strahlung
  \end{itemize}

  Für Personen die im Beruf an Strahlung ausgesetzt sind, ist das schweizerische Grenzwert 20mSv.
  Ausserdem gilt, dass jegliche Strahlung einen Nutzen haben muss, und diese Strahlung so tief wie möglich sein soll.
\end{document}    