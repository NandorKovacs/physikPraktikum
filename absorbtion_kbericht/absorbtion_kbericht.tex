\documentclass[12pt, a4paper, twoside]{article}
\usepackage{pdfpages}
\usepackage{../labreport}

\setlabreportopts[authors={Nandor Kovacs \& Céline Schuster},
    title={Mathematisches Pendel},
    subtitle={Untersuchung der Eigenschaften eines Mathematischen Pendels},
    date={\today},
    labdate={10. März 2022}
]

\begin{document}
  \section{Versuch 2} 
  \subsection{}
  \datadiagram{  
    enlargelimits=0.2,
    ylabel={Event Anzahl},
    xlabel={Bleiplattendicke in mm}
  }
  {
    \addplot+[color=red] table[x expr=\thisrow{mm}, y expr=\thisrow{e} - 38, col sep=comma] {messungen.csv};
    \addplot+[color=yellow, domain=0:21.5, mark=none]{347.5};
  }

  Bei 0mm beträgt die Anzahl von Events im Zeitrahmen 695.
  Die Hälfte davon ist 347.5. Die gelbe Linie zeichnet diesen Wert ein.
  Nach unseren Daten ist der Halbwertswert etwa bei 12.5mm. 

  \subsection{}
  
  \section{Versuch 3}
\end{document}    